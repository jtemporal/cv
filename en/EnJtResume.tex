%!TEX TS-program = xelatex
\documentclass[]{friggeri-cv}
\addbibresource{bibliography.bib}

\begin{document}
\header{Jessica}{Temporal}
       {Python Developer, Data Scientist and Bachelor in Biomedical Informatics}


% In the aside, each new line forces a line break
\begin{aside}
  \section{about}
    Sao Paulo, Brazil
    +55 16 99229-8700
    ~
    \href{mailto:jessicatemporal@gmail.com}{jessicatemporal at gmail.com}
    \href{http://jtemporal.com}{jtemporal.com}
    \href{http://github.com/jtemporal/}{github: jtemporal}
  \section{languages}
    bilingual portuguese/english
  \section{programming}
    Python
    R
    Java, C, C++
\end{aside}

\section{experience}

\begin{entrylist}
  \entrysecondtype
    {02–07 2009}
    {Serenata de Amor Operation}
    {Data Science Brigade}
    {Data Scientist}
    {Serenata de Amor Operation promotes transparency using data science. It's an open source and open data project with great traction in the tech community. The project is named after the Toblerone Affair, a Swedish political scandal, in which a politician has been forced to withdrew her candidacy for Prime Minister after found using a government credit card for buying a chocolate bar. Aiming that same accuracy level we leverage public and independently collected databases to estimate the probability of an expense being suspect, combining Machine Learning, Statistics, Data Science and white hacking skills to cause a positive impact
      \begin{itemize}
        \item Succeeded in making Brazilian politicians to return improperly spent public money
        \item 216 politicians reported for the possibly illegal use of 378k Brazilian reais (123.5k USD) in the first week of reports
        \item Largest technology crowdfunded project in Brazil
        \item Open Source project that uses Python, Jupyter Notebooks, Django, Docker to develop an Artificial Intelligence and it's Twitter Bot
      \end{itemize}
    }
  \entrysecondtype
    {02–07 2009}
    {Administrative Software}
    {Hippo Drs}
    {Python Developer Intern}
    {Hippo Drs is a technology in health care start up. Won first place in the Business Maraton at Campus Party Brazil 2016. As an intern, I was responsible for the development and implementation of Hippo Drs platform functionalities. Initially as full-stack developer and later more focused on back-end tasks
      \begin{itemize}
        \item Hippo Drs. project involved the use of Python, Django, MySQL, NGINX, API Rest, Git, Linux and automated deployment and tests
      \end{itemize}
    }
\end{entrylist}

\section{technologies and interests}

\small Data Science, Pandas, NumPy, Big Data, Open Source, VIM, Git, Linux, MySQL, PostgreSQL, Django, REST APIs, TDD, Machine Learning, Docker, PySpark, Jekyll

\section{education}

\begin{entrylist}
  \entrysecondtype
    {02-2011 02-2016}
    {Bachelor in Biomedical Informatics}
    {Ribeirao Preto, SP - Brazil}
    {University of Sao Paulo}
    {Inter-unit course School of Medicine of Ribeirão Preto and School of Philosophy, Sciences and Literature of Ribeirão Preto aimed to train professionals in three major areas: Bioformatics, Medical Images Processing and Medical Administrative Systems
      \begin{itemize}
        \item Undergraduate research student in Natural Language Processing - 2016
        \item Biomedical Informatics Student Body President - 2014
        \item Biomedical Informatics Student Body Director of communications - 2013
        \item Volunteer at the X Bioinformatics Summer Course - 2014
      \end{itemize}
    }
\end{entrylist}

\begin{entrylist}
  \entrysecondtype
    {06-2015 01-2016}
    {Bioinformatics research student}
    {Ribeirao Preto, SP - Brazil}
    {Omics Laboratory - University of Sao Paulo}
    {\begin{itemize}
        \item Participated as data scientist in STEM Cells research
        \item Analysed microRNA expression in meningioma cancer cells
        \item The projects mainly used R, cloud computing, Linux and Git as resources
      \end{itemize}
    }
  \entrysecondtype
    {06-2015 01-2016}
    {Bioinformatics research student}
    {Ribeirao Preto, SP - Brazil}
    {Hematology Laboratory - Hospital das Clinicas}
    {
      \begin{itemize}
        \item Co-author in a Role of TAM in post-sepsis tumor progression paper published on January 2016, mainly contributing by developing scripts to analyse gene expression data
        \item Bioinformatics summer internship at Weill Cornell Medical - July 2015
      \end{itemize}
    }
\end{entrylist}


\section{projects}

\begin{entrylist}
  \entry
    {09-2016 02-2017}
    {PyLadies}
    {Ribeirao Preto, SP - Brazil}
    {Helped initiate a PyLadies chapter in Ribeirao Preto city}
  \entrysecondtype
    {07-2016 12-2016}
    {Jessie - NLP pipeline}
    {Ribeirao Preto, SP - Brazil}
    {\href{http://jtemporal.com/jessie/}{jtemporal.com/jessie/}}
    {Jessie is my final graduation project, it's main objective is to analyse and make public a collection of Portuguese tweets that mentioned four diseases with major impact on the Brazilian health scenery
      \begin{itemize}
        \item 2 trained POS-taggers were made available
        \item Validation of a base method natural language processing with tweets in Portuguese
      \end{itemize}
    }
\end{entrylist}

\section{talks}

\begin{entrylist}
  \talkentry
    {10-2017}
    {Serenata de Amor: Inteligência artificial usando dados abertos governamentais}
    {\href{http://2017.pythonbrasil.org.br/}{2017.pythonbrasil.org.br}}
    {Python Brasil - Belo Horizonte, MG - Brazil}
    {lorem ipsum}
  \talkentry
    {09-2017}
    {Serenata de Amor: Tracking Congress People's Expense Claims in Brazil}
    {\href{http://civictechfest.org/}{civictechfest.org/}}
    {CivicTechFest/TicTeC@Taipei Taipei - Taiwan}
    {lorem ipsum}
\end{entrylist}

\section{volunteer}

\begin{entrylist}
  \entry
    {2016 - 2017}
    {Organizing Comitee Member}
    {pug Ribeirao Preto, SP - Brazil}
    {In 2016, Caipyra was a four days Python-'focused' event, with other subjects and languages appearing in talks as event 'guests', in the spirit of welcoming different development tools and practices to the community. In the second edition (2017) Caipyra grew and had a Django Girls edition as pre-event. \footnotesize\href{2016.caipyra.python.org.br}{2016.caipyra.python.org.br}}
  \entry
    {10-2016}
    {Coach}
    {Django Girls Florianopolis, SC - Brazil}
    {Django Girls is a free, one day event, organized by volunteers  from the Django Girls community. At the event I was responsible for three atendees of the Florianopolis edition}
\end{entrylist}

\end{document}
